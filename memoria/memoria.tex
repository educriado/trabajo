\begin{document}

\section{Introducci\'on}
El objetivo de esta segunda parte del TP6 de la asignatura de Inteligencia
Artificial es el desarrollo de un detector de spam (correo basura) usando
un clasificador de Bayes ingenuo.

Para ello vamos a utilizar la base de datos p\'ublica Enron-Spam, que contiene
correos electr\'onicos preprocesador para realizar este tipo de an\'alisis.
El lenguaje para la implementaci\'on es \textit{Python} y vamos a hacer uso del
paquete \textit{scikit\_learn} para el aprendizaje del clasificador.
Por \'ultimo, para dibujar alguna gr\'afica de los resultados tambi\'en vamos
a utilizar el paquete matplotlib.

\section{Desarrollo de la pr\'actica}

\subsection{Entrenamiento del clasificador}
El procedimiento para el entrenamiento del clasificador consiste en seguir
determinados pasos:
\begin{itemize}
	\item Creaci\'on de la estructura de bolsas de palabras:
	\item Llenado de la estructura con las cuentas:
	\item Creaci\'on de la estructura con la frecuencia de las palabras:
	\item C\'alculo de las frecuencias de las palabras:
	\item Selecci\'on de un m\'etodo de aprendizaje para el clasificador:
	para ello tenemos que elegir un valor de suavizado de Laplace adecuado.
	Este lo calculamos antes y en este caso con unas divisiones en cinco del
	conjunto de correos ha sido el valor 1 siempre el \'optimo.
	\item Predicci\'on de los resultados utilizando los correos de test:
\end{itemize}

\subsection{Evaluaci\'on del clasificador}
% incluir todas las imagenes
Para comparar las distintas configuraciones posibles se utilizan tres
m\'etricas:
\begin{itemize}
	\item Matriz de confusi\'on:
	\item Curva precisi\'on recall:
	\item F1 score:
\end{itemize}
Los resultados que hemos obtenido han sido los siguientes:
\begin{verbatim}
Resultados con clasificador Multinomial, alpha = 1
==================================================
Porcentaje de fallos:  0.998185117967 %
Porcentaje de aciertos:  99.001814882 %
Matriz de confusion:
[[794   9]
 [  6 293]]
F1 score:
0.975041597338

Resultados con clasficador Bernoulli, alpha = 1
==================================================
Porcentaje de fallos:  1.17967332123 %
Porcentaje de aciertos:  98.8203266788 %
Matriz de confusion:
[[786  17]
 [  4 295]]
F1 score:
0.965630114566

Resultados con clasificador Multinomial, alpha = 1, usando bigramas
===================================================================
Porcentaje de fallos:  1.3611615245 %
Porcentaje de aciertos:  98.6388384755 %
Matriz de confusion:
[[791  12]
 [  5 294]]
F1 score:
0.971900826446

Resultados con clasificador Bernoulli, alpha = , usando bigramas
===================================================================
Porcentaje de fallos:  4.17422867514 %
Porcentaje de aciertos:  95.8257713249 %
Matriz de confusion:
[[754  49]
 [  1 298]]
F1 score:
0.922600619195
\end{verbatim}

\section{Anexo: K-fold cross validation}
% si da espacio y TIEMPO explicar el algoritmo; puede ser bueno para rellenar
\end{document}
